\section{Exploratory Data Analysis Dependence}

% 1-1
%%%%%%%%%%%%%%%%%%%%%%%%%%%%%%%%%%%%%%%%
\frame{
\frametitle{Quantlet: Exploratory Data Analysis: Dependence}
\begin{itemize}
\item aim of this quantlet: show relations between variables, especially the target variable SalePrice 
\item producing different graphical representations
\begin{itemize}
\item correlation matrix	
\item barplot  
\item boxplots 

\end{itemize}

\end{itemize}
}

% 1-2
%%%%%%%%%%%%%%%%%%%%%%%%%%%%%%%%%%%%%%%%
\begin{frame}[fragile]
\frametitle{Correlation Matrix of all numeric variables: Code}

\begin{lstlisting}[basicstyle=\tiny][language=R]
corr.func = function(data, cut.value, corr.mat = FALSE, corr.test = FALSE, significance = 0.05) {
	corr.numeric =cor(na.omit(numeric.data)) 
	find.rows = apply(corr.numeric, 1, function(x) sum(abs(x)> 					abs(cut.value))>1)
	corr.numeric.adjusted = corr.numeric[find.rows, find.rows]
	low.corr = colnames(corr.numeric) %in%										colnames(corr.numeric.adjusted)

    pdf("Corrplot.pdf")
    if (corr.test == FALSE) {
        corrplot(corr.numeric.adjusted, method = "square")
    } else {
        corrplot(corr.numeric.adjusted, 
        p.mat =	correlation.test(corr.numeric.adjusted),
        sig.level = ignificance, method = "square")
    }
    dev.off()
    
    if (corr.mat == TRUE) 
        return(corr.numeric.adjusted)
  }

\end{lstlisting}

\end{frame}
%1-3
%%%%%%%%%%%%%%%%%%%%%%%%%%%%%%%%%%%%%%%%
\begin{frame}[fragile]
\frametitle{Correlation Matrix of all numeric variables: Code ctd.}
\begin{lstlisting}[basicstyle=\tiny][language=R]
correlation.test = function(corr.data) {
    corr.data            = as.matrix(corr.data)
    n                    = ncol(corr.data)
    p.value.matrix       = matrix(NA, n, n)
    diag(p.value.matrix) = 0
    
    for (i in 1:(n - 1)) {
     for (j in (i + 1):n) {
      tmp = cor.test(corr.data[, i], corr.data[, j])            
      p.value.matrix[i, j] = p.value.matrix[j, i] = tmp$p.value 
            }
      colnames(p.value.matrix) = rownames(p.value.matrix) =          		  colnames(corr.numeric.adjusted))
        }
	return(p.value.matrix)
    }
\end{lstlisting}

\end{frame}
% 1-4
%%%%%%%%%%%%%%%%%%%%%%%%%%%%%%%%%%%%%%%%
\frame{
\frametitle{Plot of Correlations (Corrplot)}

\begin{columns}
\begin{column}{0.5\textwidth}
\includegraphics[width=\textwidth,keepaspectratio]{\string"../quantlets/Exploratory_Data_Analysis_Dependence/Corrplot\string".pdf}
\end{column}
\begin{column}{0.5\textwidth}
\begin{itemize}
\item A cut-off value of 0.3 was used: Variables with no correlation over 0.3 do not show up in the plot
\item A significance level of 0.5 was used to test the correlations (crosses indicate no significance)
\end{itemize}
\end{column}
\end{columns}
}

% 1-5
%%%%%%%%%%%%%%%%%%%%%%%%%%%%%%%%%%%%%%%%
\begin{frame}[fragile]
\frametitle{Barplot: Code}

\begin{lstlisting}[basicstyle=\tiny][language=R]
corr.barplot = function(numb.corr = 36) {
        correlation.vars    = names(numeric.data) %in% c("SalePrice")
        correlation.data    = numeric.data[!correlation.vars]                  
        correlations        = vector(length = 										length(names(correlation.data)))
        names(correlations) = names(correlation.data)        
        for (i in names(correlation.data)) {                                   
            correlations[i] = cor(numeric.data$SalePrice, 							correlation.data[i], use = "pairwise.complete.obs")}        
        
        y.plotting = correlations[order(abs(correlations), 
           decreasing = TRUE)][1:numb.corr]
        x.plotting = names(y.plotting)
        names(y.plotting) = NULL
        df = data.frame(x.plotting, y.plotting)
        df$x.plotting = factor(df$x.plotting, levels = 
           df[order(df$y.plotting, decreasing = TRUE), "x.plotting"])
        
        ggplot(data = df, aes(x.plotting, y.plotting), fill = as.factor(x.plotting)) + geom_bar(stat = "identity") + 
            theme(axis.title.x = element_blank(), axis.text.x = element_text(angle = 90, vjust = 0.5, size = 10)) + 
            ylab("Correlation") + ggtitle(paste("Barplot of the", numb.corr, "highest bivariate correlations with SalePrice", 
            sep = " "))
}

\end{lstlisting}

\end{frame}

% 1-5
%%%%%%%%%%%%%%%%%%%%%%%%%%%%%%%%%%%%%%%%
\frame{
\frametitle{Ordered barplot for the correlations with SalePrice}

\begin{columns}
\begin{column}{0.5\textwidth}
%\includegraphics[width=\textwidth,keepaspectratio]{\string"../quantlets/Exploratory_Data_Analysis_Dependence/Barplot_ordered\string".pdf}
\end{column}
\begin{column}{0.5\textwidth}
\begin{itemize}
\item The plot shows the 20 highest correlated numeric variables
\item The few negative correlations are not plotted, since they are very low 
\end{itemize}
\end{column}
\end{columns}
}


% 1-5
%%%%%%%%%%%%%%%%%%%%%%%%%%%%%%%%%%%%%%%%
\begin{frame}[fragile]
\frametitle{Boxplot: Code}

\begin{lstlisting}[basicstyle=\tiny][language=R]
boxplot.target = function(categoric) {
        categoric.x = data[, categoric]
        plot.data = as.data.frame(cbind(data$SalePrice, categoric.x))
        plot.data[[2]] = as.factor(plot.data[[2]])
        levels(plot.data[[2]]) = levels(categoric.x)
        
        ggplot(plot.data, aes(x = categoric.x, y = V1)) +			geom_boxplot() + 
          labs(title = paste("Boxplots of SalePrice", 
            "\n", "depending on", categoric, sep = " "), 
             x = categoric, y = "SalePrice")
}

\end{lstlisting}
\end{frame}

% 1-6
%%%%%%%%%%%%%%%%%%%%%%%%%%%%%%%%%%%%%%%%
\frame{
\frametitle{Boxplots of SalePrice}

\begin{columns}
\begin{column}{0.5\textwidth}
\includegraphics[width=\textwidth,keepaspectratio]{\string"../quantlets/Exploratory_Data_Analysis_Dependence/boxplot_1through6\string".pdf}
\end{column}
\begin{column}{0.5\textwidth}
\begin{itemize}
\item Example of boxplots for SalePrice based on the levels of categorical variables
\item Differences in median and spread across levels indicate a possibly good predictor variable 
\end{itemize}
\end{column}
\end{columns}
}





