\section{Conclusions}
In this project we implemented different regression based models, a gradient boosting machine as well as a random forest model to predict the sale price of houses and compare the predictive performance of the different models. 
While all models show relatively high explanatory power on both test and training data, we find that the stepwise algorithms and the gradient boosting machine show the best predictive performance

With regards to these results one can reach several conclusions. First the stepwise linear models show the best results with respect to both the mean squared error and the mean absolute error. This suggests that the relationship between the predictor variables and the log price do indeed seem to be linear, which in turn implies a multiplicative relationship with the Sale Price. A possible extension of our work to verify this hypothesis would consist of repeating the analysis for prices without the log transformation.The fact that both Lasso and Ridge Regression are outperformed by the unregularised models suggests that we do not need regularisation as a measure against overfitting. The picture looks very different with regards to the Random Forest model however. Here we observe a very low error on the training set with a rather high one on the test set, which suggests possible overfitting. Overall the Random Forest model compares poorly with both the stepwise linear models and the Gradient Boosting Machine, which suggests that this is not a particular viable approach in this case. At this point we would like to remind the reader that the focus of this project has not been tuning one model to maximise the prediction accuracy, but giving a relative quick overview of the different models that are available as well as demonstrating methods for data preprocessing. Therefore the performance of these modelling approaches might differ significantly if more work would be invested in parameter tuning and variable selection.
Apart from these findings regarding the relative performance of different models this project has also served to once again demonstrate the versatile and powerful modelling capabilities of the \textit{R} programming language as well as its easy to use interface. Apart from the packages used for modelling and tuning the easy production of graphical output using \textit{ggplot2}  as well as the integration of tabular output in latex using \textit{xtable} have been especially useful. 