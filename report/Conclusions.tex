\section{Conclusions}
With regards to the value of the analysed modelling approaches for predicting this dataset one can reach several conclusions based on the results presented in the previous section. First the simple linear model (\textit{lm}) and the linear model with forward selection (\textit{fwd}) show the best results with respect to both the mean squared error and the mean absolute error. This suggests that the relationship between the predictor variables and the log price do indeed seem to be linear, which in turn implies a multiplicative relationship with the Sale Price. A possible extension of our work to verify this hypothesis would consist of repeating the analysis for prices without the log transformation.The fact that both Lasso and Ridge Regression are outperformed by the unregularised models suggests that we do not need regularisation in this case and there does not seem to be over fitting for linear models. The picture looks very different with regards to the Random Forest model. Here we observe a very low error on the training set with a rather high one on the test set, which suggests possible overfiting. Overall the Random Forest model compares poorly with both the unregularised linear models and the Gradient Boosting Machine, which suggests that this is not a particular viable approach in this case. At this point we would like to remind the reader that the focus of this project has not been tuning one model to maximise the prediction accuracy, but giving a relative quick overview of the different models that are available as well as demonstrating methods for data preprocessing. Therefore the performance of these modelling approaches might differ significantly if more work would be invested in parameter tuning and variable selection.
Apart from these findings regarding the relative performance of different models this project has also served to once again demonstrate the versatile and powerful modelling capabilities of the \textit{R} programming language as well as its easy to use interface. Apart from the packages used for modelling and tuning the easy production of graphical output using \textit{ggplot2}  as well as the integration of tabular output in latex using \textit{xtable} have been especially useful. 