\subsection{Exploratory Data Analysis}
In this section we will present the results of the exploratory data analysis on the housing data. First we will present the results with regards to the distribution of the target variable and afterwards we will analyse the input variables as well as the dependence / correlation structure between those and the target variable. 
\subsubsection{Exploratory Data Analysis Univariate}
In Figure \ref{fig:pricehist} we have visualised the distribution of the target variable \textit{SalePrice} in the dataset. As we can see from the histogram bars as well as the empirical density plotted in red, the distribution is highly skewed with a long tail to the right. If we compare this with a normal distribution with the same mean and variance, whose density is plotted in blue, we can see large differences and confidently assume that the target variable does not follow a normal distribution. 
In Figure \ref{fig:logpricehist} we repeat this analysis after applying a log-transformation to the sale prices. In this case a very different picture emerges with the empirical density closely resembling that of a normal distribution. This picture is supported by the QQ-Plot shown in Figure \ref{fig:stdpriceqq}. Here we have also standardised the log sale prices by subtracting the mean and dividing by its standard deviation. The resulting quantiles lie very close to the line that would result from a standard normal distribution, which supports the assumption that the log sale prices are indeed normally distributed. This is of special interest since we will use this transformation in our modelling approaches, mainly to convert additive into multiplicative models. 

In Tables \ref{tab:categoric.overview1} to \ref{tab:numeric.overview2} we show the overview tables explained in the implementation section of this analysis. From the tables we can see that only very few variables contain any NAs. We can also see that there are quite a few categoric variables where the vast majority of observations takes on the same factor level (for example \textit{Street, RoofMat1, LandSlope}), we will explain in the sections below how we tackle that challenge.

\begin{figure}[H]
  \centering
\includegraphics[width=0.65\textwidth,keepaspectratio]{\string"../quantlets/SPL_Exploratory_Data_Analysis/PriceHist\string".png}
  \caption{Histogram of SalePrice}\label{fig:pricehist}
\end{figure}

\begin{figure}[H]
  \centering
\includegraphics[width=0.65\textwidth,keepaspectratio]{\string"../quantlets/SPL_Exploratory_Data_Analysis/LogPriceHist\string".png}
  \caption{Histogram of log SalePrice}\label{fig:logpricehist}
\end{figure}

\begin{figure}[H]
  \centering
\includegraphics[width=0.65\textwidth,keepaspectratio]{\string"../quantlets/SPL_Exploratory_Data_Analysis/StdPriceQQ\string".png}
  \caption{QQ-Plot of standardised log SalePrice}\label{fig:stdpriceqq}
\end{figure}




\subsubsection{Exploratory Data Analysis Dependence}
Regarding the analysis of the dependence structures we get different graphical representations depending on the scale of the variables. First looking at Figure \ref{fig:corrgram} one can see that 27 out of 37 numeric variables in the dataset are present in the correlogram. This is the case, because a cutoff value of 0.3 was chosen in order not to overcrowd the plot with all variables, making it harder to spot potential patterns. It stands out, that the plot shows more positive (blue) than negative (red) correlations. Additionally to the colour coding of the correlation's sign, one can see black crosses indicating, where the corresponding test does not reject the null hypothesis of no correlation. Overall more correlations are insignificant than significant, but looking at the dependent variable of this analysis the picture changes. Most of the correlations are significant and positive, which suggests that the numeric variables in the dataset can be good predictors in our models. To examine the relationship between \textit{SalePrice} and the numeric variables further we plot the correlations in an ordered barplot. Figure \ref{fig:barplot_ordered} shows that all of the 20 highest correlations are positive. Only 10 of those have a value above 0.5. The plot reveals, that many of the numeric variables in the dataset are not very strongly correlated with \textit{SalePrice}. The variables with the highest correlations are \textit{OverallQual}, \textit{GrLivArea}, \textit{GarageCars}, \textit{GarageArea} and \textit{TotalBsmtSF}. This result is not surprising, since all of those are in context of either living and garage space or overall material and finish of the houses.

Besides the numeric variables, the dataset contains 43 categoric variables, which can prove useful in predicting house prices. In order to analyse the relationships in the data, we create boxplots of the target variable for each level of the independent variables. If the distribution of \textit{SalePrice} is very different for each level, the boxplots signal information that can be used in building models. Figures \ref{fig:box7to12}, \ref{fig:box37to43} and Figures  \ref{fig:box1to6} through \ref{fig:box31to36} present the results of this analysis (last five in Appendix). Variables like \textit{Neighbourhood} in Figure \ref{fig:box7to12} show big changes in the distribution of the target variable across factor levels. Other variables like \textit{GarageCond} and \textit{PavedDrive} in Figure \ref{fig:box37to43} look less promising. In general though, there seems to be some dependence of the distribution of the target variable for most categoric variables, which suggests, that the input variables are  linked to \textit{SalePrice} strongly enough to be able to produce relatively accurate predictions. 
\begin{figure}[H]
  \centering
\includegraphics[width=0.8\textwidth,keepaspectratio]{\string"../quantlets/SPL_Exploratory_Data_Analysis_Dependence/Corrplot\string".png}
  \caption{Visualization of correlation structure for variables with higher dependencies}\label{fig:corrgram}
\end{figure}

\begin{figure}[H]
  \centering
\includegraphics[height=10cm,keepaspectratio]{\string"../quantlets/SPL_Exploratory_Data_Analysis_Dependence/Barplot_ordered\string".png}
  \caption{Barplot of correlations with \textit{SalePrice} (ordered from high to low)}\label{fig:barplot_ordered}
\end{figure}

\begin{figure}[H]
\centering
\includegraphics[height=10cm,keepaspectratio]{\string"../quantlets/SPL_Exploratory_Data_Analysis_Dependence/boxplot_7through12\string".png}
\caption{boxplots of \textit{SalePrice} for levels of categoric variables 7-12}
\label{fig:box7to12}
\end{figure}

\begin{figure}[H]
\centering
	\includegraphics[height=10cm,keepaspectratio]{\string"../quantlets/SPL_Exploratory_Data_Analysis_Dependence/boxplot_37through43\string".png}
  	\caption{boxplots of \textit{SalePrice} for levels of categoric variables 37-43}
  	\label{fig:box37to43}
\end{figure}




