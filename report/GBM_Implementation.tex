\subsection{Gradient Boosting Machines}
A second alternative to the linear models that we tested is the Gradient Boosting Machine. The implementation of this model in R is very similar to that of the Random Forest. Again we use the \textit{caret} package to tune the model using the same Cross Validation as in the tuning of the Random Forest. Therefore the only two differences in the call to the \textit{train} function lie in specifying "gbm" as the \textit{method} parameter and choosing a different tuning grid. In this case we have to specify values for the four tuning parameters \textit{n.trees, shrinkage, n.minobsinnode, interaction.depth}. While we fix the latter two to just one level we specify three different values for \textit{n.trees} and 15 values for \textit{shrinkage} producing a grid of 45 parameter combinations.
Again we parallelise the computation to speed up the tuning process on parallel processors.

\begin{lstlisting}[language=R]
# Tuning Paramter Values for Gradient Boosting Machine
ntrees.tuning     = c(100, 500, 1000)
intdepth.tuning   = 1
shrinkage.tuning  = seq(0.01, 0.15, by = 0.01)
minobs.tuning     = 10
# Tune Gradient Boosting Machine Create Tuning Grid
gbmGrid     = expand.grid(n.trees = ntrees.tuning, interaction.depth = intdepth.tuning, shrinkage = shrinkage.tuning, n.minobsinnode = minobs.tuning)
# Start Cluster to paralellize
cl          = makeCluster(cores)
registerDoParallel(cl)
set.seed(123)
#Tune Model
gbmtuned = train(logSalePrice ~ ., data = df, method = "gbm", trControl = CrossValidation, verbose = TRUE, tuneGrid = gbmGrid)
stopCluster(cl)
\end{lstlisting}