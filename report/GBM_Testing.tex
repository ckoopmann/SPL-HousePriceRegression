\subsubsection{Gradient Boosting Machine}
Compared to the Random Forest model where we used a one dimensional tuning grid with tuning only the \textit{mtry} parameter, the tuning grid of the Gradient Boosting Machine is two dimensional with tuning across both the \textit{n.trees} and the \textit{shrinkage} parameters.
Using the caret interface one could also tune across the \textit{interaction depth} and the minimum observations in each leaf node of the trees, however in the interest of brevity and to save computational resources these parameters were set to reasonable default values. 
In Figure \ref{fig:gmb_rmse} we can see the average root mean squared errors across the cross validation for different values of both parameters. One can see a relatively high error reduction by increasing the number of trees from 100 to 500 but virtually no change by a further increase to 1000 trees. This coincides with the expectation one would have viewing the GBM as a Gradient Descent algorithm. Each new tree in this algorithm would be another step along the negative gradient towards the local minimum and the effect of additional steps on the loss function would decrease over time. Also in increase in the shrinkage parameter seems to have negative but decreasing effects on the Root Mean Squared Error. The decrease of this effect seems to be much faster for models with many trees, where from a value of $0.04$ the curve is very flat whereas in the case of 100 trees it decreases up until a value of around $0.12$ . The analysis with regards to the $R^2$ metric offers very similar results, with the graph almost looking as if it was flipped across the diagonal. 


\begin{figure}
  \centering
\includegraphics[width=0.8\textwidth,keepaspectratio]{\string"../quantlets/Random_Forest/gbm_rmse\string".png}
  \caption{GBM Tuning Results - Root Mean Squared Error}\label{fig:gmb_rmse}
\end{figure}

\begin{figure}
  \centering
\includegraphics[width=0.8\textwidth,keepaspectratio]{\string"../quantlets/Random_Forest/gbm_rsq\string".png}
  \caption{GBM Tuning Results - $R^2$}\label{fig:gbm_rsq}
\end{figure}